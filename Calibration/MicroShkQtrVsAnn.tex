 % The \commands below are required to allow sharing of the same base code via Github between TeXLive on a local machine and ShareLaTeX.  This is an ugly solution to the requirement that custom LaTeX packages be accessible, and that ShareLaTeX seems to ignore symbolic links (even if they are relative links to valid locations)
\newcommand{\econtex}{\econtexRoot/texmf-local/tex/latex/econtex}
\newcommand{\econtexSetup}{\econtexRoot/texmf-local/tex/latex/econtexSetup}
\newcommand{\econtexShortcuts}{\econtexRoot/texmf-local/tex/latex/econtexShortcuts}
\newcommand{\econtexBibMake}{\econtexRoot/texmf-local/tex/latex/econtexBibMake}
\newcommand{\econtexBibStyle}{\econtexRoot/texmf-local/bibtex/bst/econtex}
\newcommand{\notes}{\econtexRoot/texmf-local/tex/latex/handout}
\newcommand{\handoutSetup}{\econtexRoot/texmf-local/tex/latex/handoutSetup}
\newcommand{\handoutShortcuts}{\econtexRoot/texmf-local/tex/latex/handoutShortcuts}
\newcommand{\handoutBibMake}{\econtexRoot/texmf-local/tex/latex/handoutBibMake}
\newcommand{\handoutBibStyle}{\econtexRoot/texmf-local/bibtex/bst/handout}
  
\documentclass[11pt,letterpaper]{article}


\usepackage{float,amsmath,dcolumn,multicol,latexsym,ifthen,natbib,amssymb,verbatim,hyperref,vmargin,moreverb,cancel}
\usepackage{endfloat}\def\var{\operatorname{var}}
\def\std{\operatorname{std}}

\setmarginsrb{1.00in}{1.00in}{1.20in}{1.00in}{0pt}{0pt}{12pt}{24pt}
% {left}{top}{right}{bottom}{headheight}{headsep}{footheight}{footsep}


\begin{document}

\tableofcontents 

\section{Intro}
This document details the calibration of the quarterly version of the idiosyncratic
model using estimates obtained from annual idiosyncratic data.  The derivations are
verified in the associated Stata program cAndCwithStickyE{Stata}.do.

\section{Variable Definitions}

\begin{center}
\begin{tabular}{lcl}
Stata              & LaTeX                       & Description
\\ \hline
    \texttt{tShkQtr}  & $\theta_{t}$                     & transitory shock, $\log \theta\sim\mathcal{N}(-\sigma^{2}_{\theta}/2,\sigma^{2}_{\theta})$ $\Rightarrow$ $E_{t}[\theta_{t+n}]=1~\forall~n>0$
\\  \texttt{pShkQtr}  & $\psi_{t}$                     & permanent shock, $\log \psi\sim\mathcal{N}(-\sigma^{2}_{\psi}/2,\sigma^{2}_{\psi})$ $\Rightarrow$ $E_{t}[\psi_{t+n}]=1~\forall~n>0$
\\  \texttt{ltShkQtr}  & $l{\theta}_{t}$              & An $\texttt{l}$ in front of a variable indicates the log
\\  \texttt{lpShkQtr}  & $l{\psi}_{t}$                     & 
\\  \texttt{tDevQtr}  & $\theta^{\nabla}_{t}=\theta_{t}-1$    & additive form of transitory income shock in quarter $t$ %satisfying $E_{s}[\theta_{s+n}]=1~\forall~n>0$
\\  \texttt{pDevQtr}  & $\psi^{\nabla}_{t}=\psi_{t}-1$    & additive form of permanent income shock in quarter $t$ %satisfying $E_{s}[\theta_{s+n}]=1~\forall~n>0$
\\  \texttt{pLevQtr}  & $\mathbf{p}_{t}$                 & permanent income at a quarterly rate
\\  \texttt{tLevQtr}  & $\mathbf{t}_{t}$                 & transitory income at a quarterly rate (=$\theta$)
\\  \texttt{yLevQtr}  & $\mathbf{p}_{t}\theta_{t}=\mathbf{p}_{t}(1+\theta^{\nabla}_{t})$   & actual income in quarter $t$
\\  \texttt{pLevAnn}  & $\vec{\mathbf{p}}_{t}$                & permanent income for four quarters beginning at $t$ eqn \eqref{eq:pLevAnn}
\\  \texttt{tLevAnn}  & $\vec{\theta}_{t}$                & transitory income for four quarters beginning at $t$ eqn \eqref{eq:tShkAnn}
\\  \texttt{tDevAnn}  & $\vec{\theta}_{t}-1$              & transitory shock for four quarters beginning at $t$ eqn \eqref{eq:tShkAnn}
\\  \texttt{yLevAnn}  & $\vec{\mathbf{y}}_{t}$ & income for four quarters beginning at $t$
\\  \texttt{tShkAnn} & $\vec{\theta}_{t}$                  & transitory shock to income over year beginning at $t$ 
\\  \texttt{pShkAnn} & $\vec{\psi}_{t}$                  & permanent shock to income over year beginning at $t$ 
\\  \texttt{stdltShkAnn} & $\sigma_{\vec{\theta}}$ & standard deviation of transitory component of annual income
\\  \texttt{stdlpShkAnn} & $\sigma_{\vec{\mathbf{\psi}}}$ & standard deviation of permanent component of annual income
\\  \texttt{stdltShkQtr} & $\sigma_{\theta}$ & standard deviation of transitory component of quarterly income
\\  \texttt{stdlpShkQtr} & $\sigma_{\mathbf{\psi}}$ & standard deviation of permanent component of quarterly income
\end{tabular}
\end{center}

\section{Relationship Between Annual and Quarterly Transitory Shocks}

Start by assuming that the shocks to permanent income occur only once every four quarters, 
so that, e.g., 
\begin{eqnarray}
 \mathbf{p}_{t} & = & \mathbf{p}_{t+1}=\mathbf{p}_{t+2}=\mathbf{p}_{t+3} 
\end{eqnarray}
which implies
\begin{eqnarray}
   \texttt{pLevAnn} = \vec{\mathbf{p}}_{t} & = & 4\mathbf{p}_{t}  \label{eq:pLevAnn}
\\ \texttt{yLevAnn} = \vec{\mathbf{y}}_{t} & = & \mathbf{p}_{t}(1+\theta^{\nabla}_{t})+\mathbf{p}_{t+1}(1+\theta^{\nabla}_{t+1})+\mathbf{p}_{t+2}(1+\theta^{\nabla}_{t+2})+\mathbf{p}_{t+3}(1+\theta^{\nabla}_{t+3})
\\   & = & \mathbf{p}_{t}\left(4+\theta^{\nabla}_{t}+\theta^{\nabla}_{t+1}+\theta^{\nabla}_{t+2}+\theta^{\nabla}_{t+3}\right)
\\   & = & \vec{\mathbf{p}}_{t}\left(1+(\theta^{\nabla}_{t}+\theta^{\nabla}_{t+1}+\theta^{\nabla}_{t+2}+\theta^{\nabla}_{t+3})/4\right)
\\ \vec{\theta}_{t} & = & \left(1+(\theta^{\nabla}_{t}+\theta^{\nabla}_{t+1}+\theta^{\nabla}_{t+2}+\theta^{\nabla}_{t+3})/4\right) \label{eq:tShkAnn}
\end{eqnarray}

\begin{eqnarray}
    \log \vec{\theta}_{t} & \approx & (\theta^{\nabla}_{t}+\theta^{\nabla}_{t+1}+\theta^{\nabla}_{t+2}+\theta^{\nabla}_{t+3})/4
\\  \var(\log \vec{\theta}) & \approx & \var\left((\theta^{\nabla}_{t}+\theta^{\nabla}_{t+1}+\theta^{\nabla}_{t+2}+\theta^{\nabla}_{t+3})/4\right)
\end{eqnarray}
but since $\var(\theta^{\nabla}) \approx \sigma_{\theta}^{2}$, this implies
\begin{eqnarray}
  \sigma^{2}_{\vec{\theta}} & = & 4 \sigma^{2}_{\theta}/16
\\ \hat{\sigma}^{2}_{\theta} & = & 4\sigma^{2}_{\vec{\theta}}   \label{eq:varltShkQtr}
\end{eqnarray}

\section{Permanent Quarterly and Annual Shocks}
Now let's relax the assumption that permanent shocks arrive only once a year, but
(to keep the algebra simple), suppose we are looking at semiannual rather than quarterly data.

\begin{eqnarray}
  \vec{\mathbf{p}}_{t} & = & {\mathbf{p}}_{t}+{\mathbf{p}}_{t}\psi_{t+1}
\\ & = & {\mathbf{p}}_{t}(1+\psi_{t+1})
\\ & = & {\mathbf{p}}_{t}(1+(1+l{\psi}_{t+1}))
\\ & = & 2{\mathbf{p}}_{t}(1+l{\psi}_{t+1}/2)
\\ \vec{\mathbf{p}}_{t+2} & = & {\mathbf{p}}_{t}\psi_{t+1}\psi_{t+2}(1+\psi_{t+3})
\\  & = & 2{\mathbf{p}}_{t}(1+l{\psi}_{t+1})(1+l{\psi}_{t+2})(1+l{\psi}_{t+3}/2)
\\ l\vec{\mathbf{p}}_{t+2} & \approx & \log 2{\mathbf{p}}_{t} + l{\psi}_{t+1}+l{\psi}_{t+2}+l{\psi}_{t+3}/2
\\ l\vec{\mathbf{p}}_{t+2} - l\vec{\mathbf{p}}_{t} & \approx & l{\psi}_{t+1}+l{\psi}_{t+2}+l{\psi}_{t+3}/2-l{\psi}_{t+1}/2
\\ & = & l{\psi}_{t+1}/2+l{\psi}_{t+2}+l{\psi}_{t+3}/2
\\ \var(l\vec{\mathbf{p}}_{t+2} - l\vec{\mathbf{p}}_{t}) & \approx & (1+1/2) \var(l{\psi})
\end{eqnarray}

Similarly,
\begin{eqnarray}
    l\vec{\mathbf{p}}_{t+4} & \approx & \log 2{\mathbf{p}}_{t} + l{\psi}_{t+1}+l{\psi}_{t+2}+ l{\psi}_{t+3}+l{\psi}_{t+4}+l{\psi}_{t+5}/2
\\  l\vec{\mathbf{p}}_{t+4}-l\vec{\mathbf{p}}_{t} & \approx & l{\psi}_{t+1}/2+l{\psi}_{t+2}+ l{\psi}_{t+3}+l{\psi}_{t+4}+l{\psi}_{t+5}/2
\\ \var(l\vec{\mathbf{p}}_{t+4}-l\vec{\mathbf{p}}_{t}) & \approx & (3+1/2) \var(l{\psi})
\end{eqnarray}
so 
\begin{eqnarray}
  \var(l\vec{\mathbf{p}}_{t+4}-l\vec{\mathbf{p}}_{t}) - \var(l\vec{\mathbf{p}}_{t+2} - l\vec{\mathbf{p}}_{t}) & = & 2 \var(l{\psi})
\end{eqnarray}

The corresponding logic for quarterly (as opposed to semiannual) data leads to the conclusion that
\begin{eqnarray}
  \var(l\vec{\mathbf{p}}_{t+8}-l\vec{\mathbf{p}}_{t}) - \var(l\vec{\mathbf{p}}_{t+4} - l\vec{\mathbf{p}}_{t}) & = & 4 \var(l{\psi}) = \var(l{\vec{\psi}}) =\sigma^{2}_{\vec{\psi}} \label{eq:varvecp}
\end{eqnarray}

Thus, our estimate of the variance of quarterly shocks to permanent income is
\begin{eqnarray}
  \hat{\sigma}^{2}_{\psi} & = & \sigma^{2}_{\vec{\psi}} \label{eq:varlpShkQtr}
\end{eqnarray}

\begin{comment}
\section{Muth Updating Formula}

In a context similar to the one outlined here, \cite{muthOptimal} showed that a consumer who
only observes actual income but wants to know permanent income will form a rational
estimate of permanent income from 
\begin{eqnarray}
  \bar{\mathbf{p}}_{t} & = & \pi \left[ y_{t}+(1-\pi)y_{t-1}+(1-\pi)^{2}y_{t-2}+\ldots\right]
\end{eqnarray}
where
\begin{eqnarray}
  \pi & = & \left(\frac{\sigma^{2}_{\psi}}{\sigma^{2}_{\psi}+\sigma^{2}_{\theta}}\right) \label{eq:pieq}
\end{eqnarray}
\end{comment}

\section{Calibration}

The key parameters are taken from \cite{carroll&samwick:nature}, Table 1, first row:
The estimated annual variances of transitory and permanent components of income.  
For purposes of this paper, those figures are rounded off, as below.  All other
statistics are derived from these two.

\begin{center}
\input MicroShkQtrVsAnnStata.texinput
\end{center}
\begin{comment}
Note (as per the footnote of the table) that these parameters imply an implausibly
low value of $\pi$ (equivalently, an implausibly high value for the serial correlation
coefficient of consumption growth, $(1-\pi)$).
\end{comment}

 
\input econtexBibMake


\end{document}
