\cite{gabaixSparsityQJE} has recently proposed a framework that is much simpler than the full rational inattention framework of \cite{simsInattention}, but aims to capture much of its essence.  This approach is relatively new, and while it does promise to be more tractable than the full-bore Simsian framework, even the simplified Gabaix approach would be difficult to embed in a model with a standard treatment of transitory and persistent income shocks, precautionary motives, liquidity constraints, and other complexities entailed in modern models of microeconomic consumption decisions.\footnote{\cite{gabaixSparsityQJE} proposes a framework in which consumers perceive a simplified version of the world because there is a cost to paying attention.  The existence of a fixed cost of paying attention means that beliefs are not updated continuously but episodically, and the framework generates dynamics that, when aggregated, resemble partial adjustment dynamics.  It is beyond the scope of this paper (and would be an interesting project in itself) to determine how this framework would apply in a context like ours, where there are four distinct kinds of shocks (aggregate and idiosyncratic, transitory and permanent), each with very different rewards to attention.}
